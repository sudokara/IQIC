\documentclass{article}
\usepackage[utf8]{inputenc}
\usepackage{graphicx} % Required for inserting images
\graphicspath{ {./images/} }
\usepackage{mathtools} % for math tool slike arrows

\title{Introduction to Quantum Information and Computing Half 2 Lecture 7}
\author{Shrikara A, Arnav Negi, Kriti Gupta, Manav Shah, Mohammed Shamil,\\ Shiven Sinha, Swayam Agarwal, Vineeth Bhat, Yash Adivarekar}
\date{\(24^t^h\) February, 2023}



\begin{document}
\linespread{2}

\maketitle

% \section{Introduction}

So S is basically 
\begin{align*}
    2[sin(\theta/2)]|w> + cos(\theta/2)|S_{\overline{w}}>]
\end{align*}

$D=|S><S|-I$
\raggedright
\begin{align*}
% \raggedright
D &= 2[ |w><w|sin^2(\theta/2) + |w><S_{\overline{w}}|sin(\theta/2)cos(\theta/2) \\&+ |S_{\overline{w}}><w|sin(\theta/2)cos(\theta/2)+ |S_{\overline{w}}><S_{\overline{w}}|cos^2(\theta/2) ]
% \raggedright
\end{align*}

After all D is a $2\times2$ matrix \\
So in the basis {$|S_{\overline{w}}>,|w>$} D can be represented as : 
\newline \newline
D = 
\begin{bmatrix}
cos(\theta) & sin(\theta) \\
sin(\theta) & -cos(\theta)
\end{bmatrix}
\newline
We are given $G = D\cdot u_f$
\newline\newline
G = 
\begin{bmatrix}
cos(\theta) & sin(\theta) \\
sin(\theta) & -cos(\theta)
\end{bmatrix}
\times
\begin{bmatrix}
1 & 0\\
0 & -1
\end{bmatrix}
\newline \newline
G = 
\begin{bmatrix}
cos(\theta) & -sin(\theta) \\
sin(\theta) & cos(\theta)
\end{bmatrix}\\
\newline
As we can see G is a rotation matrix. \\
$|S> = G^k\cdot|S>$\\
Let's see the product one time : 

    
$G|S>$ = 
\begin{bmatrix}
cos(\theta) & -sin(\theta) \\
sin(\theta) & cos(\theta)
\end{bmatrix}$|S>$\\
$G|S>$ = sin(\theta+\theta/2)|w> + cos(\theta+\theta/2)|S_{\overline{w}}>\\
\newline
After k times : \\
$|S>$ = sin(k\theta+\theta/2)|w> + cos(k\theta+\theta/2)|S_{\overline{w}}>\\
At that time : 
$sin(k\theta+\theta/2) = 1$\\
$\implies \theta(k+1/2)=\pi/2$\\
$\implies \theta = \frac{\pi}{2k+1}$
We know that \\
$sin(\theta/2) = \sqrt{\frac{M}{N}} \approx \theta/2$\\
$2k+1=\frac{\pi\sqrt{N}}{2\sqrt{M}}$\\
$k=\frac{\pi\sqrt{N}}{2\sqrt{M}} - \frac{1}{2}$\\
$k\approx O(\sqrt{\frac{N}{M}})$\\
So finally \\
$G^k|S> = \frac{1}{\sqrt{M}} \sum_{f(|x>=1)}|x>$
\newline \newline
Now what if M is unknown :- \\
(a) Estimate 'M' before only  (Quantum Counting )\\
(b) Randomized Quantum Search 
\newline \newline
Amplification of amplitude : \\
$G^kH^{\otimes n}|0^n> = G^k[sin(\theta/2)]|w> + cos(\theta/2)|S_{\overline{w}}>]$\\
$\sqrt{p}=sin(\theta/2)$\\
To amplify this term to 1 I need $\frac{1}{\sqrt{p}}$ queries.\\
\newline
$A^{\frac{1}{\sqrt{p}}}u|0> \approx |\psi_{good}>$

\section{Modules of Quantum Computing }
1. Adiabatic Model : \\
$H(s) = (1-s)H_0+sH_k$\\
s\in[0,1]\\
\newline
2. Quantum walks \\
3. MDQC\\
4. Topological Quantum Channel \\







\end{document}
