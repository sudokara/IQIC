\documentclass{article}

\usepackage{hyperref}
\usepackage{amsmath}
\usepackage{amsfonts}

\title{Introduction to Quantum Information and Computing - Lecture 1}
\author{Shrikara A, Arnav Negi, Kriti Gupta, Manav Shah, Mohammed Shamil,\\ Shiven Sinha, Swayam Agarwal, Vineeth Bhat, Yash Adivarekar} % add contributors
\date{3rd January, 2023}

\begin{document}


    \pagenumbering{gobble}
    \maketitle
    \vfill
    \tableofcontents
    \newpage
    \pagenumbering{arabic}


    %fill
    \section{Introduction and Motivation}
        \subsection{Introduction}

        \subsection{Stern-Gerlach Experiment}

        \subsection{Shannon's Theory of Information}

        \subsection{Entropy as the Expectation of Surprise}

        \subsection{No Cloning Theorem}

        \subsection{Computation as a Subset of Information}

    \subsection{Outline of the Course}
        \begin{enumerate}
            \item Postulates
            \item Everything is a quantum channel
            \item Entanglement, Separability, Nonlocality
            \item Teleportation, No Cloning
            \item Entropy, Trace Distance
        \end{enumerate}


    \section{Finite Dimensional Hilbert Spaces}
        A \textit{d}-dimensional Hilbert space $\mathcal{H}$ ($1 \leq d < \infty$) is a complex vector space with an inner product defined on it. A vector in the Hilbert space $\mathcal{H}$ is denoted by $|\psi\rangle $. The inner product $\langle . ,. \rangle : \mathcal{H} \times \mathcal{H} \rightarrow \mathbb{C} $ has the following properties:
        \begin{itemize}
            \item \textit{Non negativity -} $\langle \psi , \psi \rangle \geq 0$ $\forall$ $|\psi \rangle \in \mathcal{H}$. $\langle \psi , \psi \rangle = 0$ if and only if $\langle \psi \rangle= 0$.
            
            \item \textit{Linearity in Second Argument -} $\langle \psi , \alpha \phi_1 + \beta \phi_2 \rangle = \alpha \langle \psi , \phi_1 \rangle + \beta \langle \psi , \phi_2 \rangle $
            
            \item \textit{Conjugate Linearity in First Argument -} $\langle \alpha \psi_1 + \beta \psi_2 , \phi \rangle = \bar{\alpha} \langle \psi_1 , \phi \rangle + \bar{\beta} \langle \psi_2 , \phi \rangle$
            
            \item \textit{Conjugate Symmetry -} $\langle \psi , \phi \rangle = \overline{\langle \phi , \psi \rangle}$
        \end{itemize}
        % continue on isomorphism, standard basis and the inner product <a,b> as a'b  
        % Sumeet Khatri and Mark Wilde 
    
    \section{Describing a Closed Physical System}
        The complete description of a closed physical system is given by its state $ | \psi \rangle $ where $ | \psi \rangle \in \mathcal{H}$ ($\mathcal{H}$ is a Hilbert Space) and norm of $ | \psi \rangle $ is 1 ($\langle \psi , \psi \rangle = 1$).
        For every state $ | \psi \rangle \in \mathcal{H}$, $\exists$ $\langle \psi |$ in the dual vector space of $\mathcal{H}$. Also, $\langle \psi | = (| \psi \rangle)^{\dagger}$.

        For $| \psi \rangle$ to represent a closed system, the Hilbert Space it belongs to must have dimension $d \geq 2$, $d \in \mathbb{N}$.


\end{document}