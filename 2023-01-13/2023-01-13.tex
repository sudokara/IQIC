\documentclass{article}
\usepackage{amsmath}
\usepackage[utf8]{inputenc}
\usepackage{bbold}

\title{Introduction to Quantum Information and Computing - Lecture 4}
\author{Arnav Negi, Kriti Gupta, Manav Shah, Mohammed Shamil, Shiven Sinha ,\\Shrikara A, Swayam Agrawal, Vineeth Bhat, Yash Adivarekar}

\date{January 13, 2023}

\begin{document}

\maketitle

\section{Observables}
\begin{itemize}
    \item Each observable can be represented by Hermitian operators.
\end{itemize}
\begin{itemize}
    \item Expectation value of an observable ${\hat{O}}$ for a quantum state $\rho$  is Tr[${\hat{O}\rho}$]
\end{itemize}
\begin{itemize}
    \item If $\rho$ is pure state {$\{|\psi\rangle\langle\psi|\}$}, then:\\\\
Tr[${\hat{O}\rho}$]=Tr[${\hat{O}|\psi\rangle\langle\psi|}$]=Tr[${\langle\psi|\hat{O}|\psi\rangle}$]=${\langle\psi|\hat{O}|\psi\rangle}$
\end{itemize}
\begin{itemize}
    \item If $\rho$ is not a pure state, i.e., $\rho$=$\sum\limits_{i}p_{i}|\psi_{i}\rangle\langle\psi_{i}|$, then:\\\\
Tr[${\hat{O}\rho}$]=Tr[${\hat{O}\sum\limits_{i}p_{i}|\psi_{i}\rangle\langle\psi_{i}|}$]=$\sum\limits_{i}p_{i}$Tr[${\hat{O}|\psi_{i}\rangle\langle\psi_{i}|}$]=$\sum\limits_{i}p_{i}\langle\psi|\hat{O}|\psi_{i}\rangle$

\end{itemize}

\section{Mixed State}
Mixed state is said to be the mixture of the different pure states.\\
Mixed states can be written as:
$\sigma=\sum\limits_{i}p_{i}\rho_{i} , where \sum\limits_{i}p_{i}=1$\\
Here $\sigma$: is the mixed state , $\rho_{i}$:pure state and $p_{i}$: probability of state $\rho_{i}$ to be present in the mixed state.
\begin{itemize}
    \item A state is pure if $\rho^2=\rho$
\end{itemize}
Example1:-
Let us consider a system which has 5 qubits of $|0\rangle$ and 5 qubits of $|1\rangle$ where $\{|0\rangle,|1\rangle\}$ are orthonormal basis. Find the mixed state of the system.\\
Answer1:- $\sigma=\frac{1}{2}\{|0\rangle\langle0|+|1\rangle\langle1|\}$\\\\
Example2:-
Let us consider a system which has 5 qubits of $|+\rangle$ and 5 qubits of $|-\rangle$ where $\{|+\rangle,|-\rangle\}$ are orthonormal basis. Find the mixed state of the system.\\
Answer2:- $\sigma=\frac{1}{2}\{|+\rangle\langle+|+|-\rangle\langle-|\}$\\\\
Note:- Both $\sigma$ obtained in the above two examples are one and the same.\\
Proof:-\\
$\sigma=\frac{1}{2}\{|0\rangle\langle0|+|1\rangle\langle1|\}=\frac{1}{2}\{|+\rangle\langle+|+|-\rangle\langle-|\}$\\\\
$\sigma$=$\frac{1}{2}(\sum\limits_{i}|i\rangle\langle i|)$=$\frac{1}{2}(\sum\limits_{i}|i\rangle\langle i|)$\\\\
$\sigma=\frac{1}{2}\mathbb{1}=\frac{1}{2}\mathbb{1}$ 

\section{Superposition}
The linear combination of states is called superposition.\\
A superpositioned state can be represented as:\\
$|\psi\rangle=\sum\limits_{i}c_{i}|\psi_{i}\rangle$,\\
where $\sum\limits_{i}c_{i}^2=1$ and $|\psi_{i}\rangle$ are the states that are being superimposed.

\end{document}
