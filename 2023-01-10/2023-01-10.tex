\documentclass{article}
\usepackage[utf8]{inputenc}
\usepackage{amsmath,amssymb}
\title{Introduction to Quantum Information and Computing - Lecture 3}
\author{Arnav Negi, Kriti Gupta, Manav Shah, Mohammed Shamil,\\ Shiven Sinha ,Shrikara A, Swayam Agrawal, Vineeth Bhat, Yash Adivarekar}
\date{\(10^t^h\) February, 2023}
\begin{document}
\maketitle
\section{Multiple Systems}
Let $H_{A}$ and $H_B$ be two system. To visualize multiple systems as a single system we use the idea of tensor product.\\
$H_A \otimes H_B$ denotes the tensor product of $H_A$ and $H_B$\\
For e.g. consider the following example : \\ \newline
\begin{bmatrix}
a_{11} & a_{12} \\
a_{21} & a_{22}
\end{bmatrix}
A = \begin{bmatrix}
a_{11}A & a_{12}A \\
a_{21}A & a_{22}A
\end{bmatrix}\\
\newline
We can easily see that
\begin{align*}
    dim(H_A \otimes H_B) = dim(H_A)\times dim(H_B)
\end{align*}
\section{Noisy Quantum Theory}
Expectation value of an observable A is 
\begin{align*}
    <A>_{\psi} = <\psi|A|\psi>
\end{align*}
If $\psi$ is not normalized then normalize it by dividing with $<\psi|\psi>$.\\
Now even if I change $|\psi>$ to $e^{\iota\phi}|\psi>$ expectation value does not changes because \\
\begin{align*}
    <\psi|e^{-\iota\phi} A e^{\iota\phi}|\psi> = <\psi|A|\psi>  \tag*{($e^{\iota\phi}$ is just a scalar)}
\end{align*}
\section{Density Operator}
A quantum state is represented by a density operator defined on a  \\
Hilbert space H. \\
\begin{center}
\( \rho : H \rightarrow H\)
\end{center}
\subsection{Requirement of Density operator}
The question arises that what's the need of representing a state in terms of density operator. \\
State vectors or wave functions $(|\psi>)$ can only represent pure states. Density operator can also represent mixed state. \\
Density operator also allows for the calculation of the probabilities of the outcomes of any measurement performed upon system, using the Born rule.
\subsection{Properties of Density operator}
\begin{itemize}
    \item \(\rho \geq 0\) \\
    This means that all the eigenvalues of $\rho$ are $\geq 0$
    \item \(\rho = \rho^\dagger\) \\
    $\rho$ should be a Hermitian-matrix
    \item $Tr(\rho) = 1$ \\
    The summation of the diagonal elements of $\rho = 1$ 
\end{itemize}
So summarising, Density operator for a system is a positive semi-definite, Hermitian operator of trace one acting on the Hilbert space of the system
\newline \newline
For any state $|\psi>$, its density operator is $|\psi><\psi|$.\\
Let's prove that $|\psi><\psi|$ follows all the property of the density operator : \\
\newline
\begin{newitem}
1. We know that $(|\psi><\psi|)^2 = |\psi><\psi|\psi><\psi| = |\psi><\psi|$ \\
So it's clear that $|\psi><\psi|$ is a pure state $\implies$ all eigenvalues are $\ge$ 0.\\ So first property is satisfied. \\
% \pagebreak

\begin{flushleft}2\end{flushleft}
\begin{align*}(|\psi><\psi|)^\dagger &= (<\psi|)^\dagger (|\psi>)^\dagger \\
    &= |\psi><\psi|
\end{align}
Hence $<\psi|\psi>$ is a hermitian operator.\\
\newline
3. \begin{align*}
    Tr(|\psi><\psi|) &= Tr(<\psi|\psi>)  \tag*{(Trace Theorem)} \\
    &=1
\end{align}
Hence, the trace of $|\psi><\psi| = 1$.\newline
All the properties are satisfied by $|\psi><\psi|.
\end{newitem}

\section{Observables}
Observables are Hermitian operators. \\
Expectation value of an observable $\hat{O}$ for a quantum state $\rho$ is  $Tr(\hat{O}\rho)$\\
If $\rho$ is pure state
\begin{align*}
Tr(\hat{O}\dot|\psi><\psi|) = <\psi|\hat{O}|\psi>
\end{align}
Given that $<\psi|\psi> = 1$

% Hello i am in new section

\end{document}
