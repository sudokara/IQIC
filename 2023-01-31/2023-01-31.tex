\documentclass{article}
\usepackage{graphicx} % Required for inserting images
\usepackage{amsmath}
\usepackage[utf8]{inputenc}
\usepackage{bbold}

\title{Lecture1 part 2}
\author{Arnav Negi, Kriti Gupta, Manav Shah, Mohammed Shamil, Shiven Sinha ,\\Shrikara A, Swayam Agrawal, Vineeth Bhat, Yash Adivarekar}
\date{January 31, 2023}

\begin{document}

\maketitle

\section{Introduction}
\begin{itemize}
    \item Quantum computing - "Natural generalization of computing".
\end{itemize}
\begin{itemize}
    \item Quantum computers are computers that obey quantum physics.
\end{itemize}


\subsection{Why quantum computing?}
\begin{itemize}
    \item Extended Church Turing Thesis (ECT): Any algorithmic process can be efficiently simulated by a probabilistic Turing machine. 
\end{itemize}
\begin{itemize}
    \item David Deutsch:-Is there a physical model of computation that violates ECT?
\end{itemize}
\begin{itemize}
    \item Computation devices built using the principles of quantum physics can offer a stronger version to their thesis.
\end{itemize}\begin{itemize}
    \item Do quantum computers that obey quantum mechanics violate ECT?
\end{itemize}\begin{itemize}
    \item Feynman asks if we can simulate quantum physics on a classical computer.
\end{itemize}\begin{itemize}
    \item Number of variables to keep track of is exponential in the size of the quantum system. For example, for a 100 electron system,${2^{100}}$ bits are needed in comparison to 100 qubits.
\end{itemize}

\section{Quantum computing in the circuit model\\ (Postulates of Quantum mechanics in action)}
\subsection{State Preparation}
\begin{itemize}
    \item Prepare the quantum computer in a given initial state\\
       ${|\psi_{0}}\rangle$=${|0\rangle^{\otimes n}}$
\end{itemize}\begin{itemize}
    \item ${|0\rangle}$=${\begin{pmatrix}
1 \\
0
\end{pmatrix}}$\\
${|0\rangle\otimes|0\rangle}$=${|0\rangle^{\otimes2}}$=${|0\rangle|0\rangle}$=${|00\rangle}$
\end{itemize}

\subsection{Evolution}
\begin{itemize}
    \item Schrodinger's equation :\\\\
    ${i\frac{d|\psi\rangle}{dt}}$=${H|\psi\rangle}$\\\\
    ${|\psi(t)\rangle}$=${e^{-iHt}|\psi(0)\rangle$
\end{itemize}
\begin{itemize}
    \item H: Hamiltonian observable for energy
\end{itemize}
\begin{itemize}
    \item ${u_t}$=${e^{-iHt}$
\end{itemize}
\begin{itemize}
    \item The initial state ${|\psi_0\rangle}$ evolves based on a series of unitary operators  \\
    ${|\psi_f\rangle}$=${u_tu_{t-1}u_{t-2}...u_2u_1|\psi_0\rangle}$\\
    where each ${u_i}$ is a quantum gate
\end{itemize}

\subsection{Measurement}

\begin{itemize}
    \item Measure the final state in the computational basis: M={${|j\rangle\langle j|}$ where ${j\in \{0,1\}^n}$}
\end{itemize}
\begin{itemize}
    \item Observe ${|f\rangle}$ with probability p such that: \\
    ${p=|\langle f|\psi_f\rangle|^2}$
\end{itemize}

\end{document}