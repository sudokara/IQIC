\documentclass{article}
\usepackage[utf8]{inputenc}

\title{IQIC - January 27rd, 2023}

\begin{document}

\maketitle

\section{Recapitulation of Quantum Channels}

A quantum channel $\mathcal{N}_{A \rightarrow B}$ taking operators from the set of bounded operators in $A$ to the set of bounded operators in $B$ is defined as

\[\mathcal{N}_{A \rightarrow B} : \mathcal{B}(A) \rightarrow \mathcal{B}(B)\]

has the following properties: 

\subsection{Trace Preserving}

\[\forall X \in \mathcal{B}(A),\quad Tr(\mathcal{N}_{A \rightarrow B}(X))=Tr(X) \]

This preserves the validity of the outcome of the operation by keeping the trace as $1$.

\subsection{Completely Positive}

Taking the maximally entangled state $\Phi_{RA}$, defined as 

\[\Phi_{RA} = | \phi \rangle \langle \phi | _{RA}\]

where the system $R$ is a space characteristic of the channel, and,

\[| \phi \rangle _{RA} = \frac{1}{\sqrt{d}}\ \Sigma^{d-1}_{i=0} | i \rangle_R | i \rangle_A \]

where 

\[ \{| i \rangle_R \}_i \quad \textrm{forms an orthonormal basis of R} \]
\[ \{| i \rangle_A \}_i \quad \textrm{forms an orthonormal basis of A} \]
\[d = infinum\{ dim(A), dim(R)\}\]

the channel produces a semi-positive definite state as output regardless of input if 

\[ \mathcal{N}_{A \rightarrow B}(\Phi_{RA}) 	\geq 0 \]

 Here, $\mathcal{N}_{A \rightarrow B}(\Phi_{RA})$ is referred to as the \emph{Choi} of the channel.

\section{Pure States}

Pure states can be represented by $| \phi \rangle _{RA}$ such as:

\[| \phi \rangle _{RA} = \Sigma^{d-1}_{i=0} \sqrt{p_i}| i \rangle_R | i \rangle_A \]

where $d$, $| i \rangle_R$ and $| i \rangle_A$ remain the same as defined in the recapitulation. 

$p_i$ represents the probability of $| i \rangle_R | i \rangle_A$ in the superposition of states, and $\Sigma^{d-1}_{i=0} {p_i} = 1$.

\noindent Further, if the number of basis vectors in $R$ exceeds $d$ then we can choose any $d$ arbitrary basis vectors for forming 
$\{| i \rangle_R \}_i$

\section{Further comments}

\begin{description}

\item It must be noted that we are working in an \emph{open quantum system} while working with quantum channels

\item Also,
\[ \mathcal{N}_{A \rightarrow B}(\varphi_A) = Tr_{_{E'}}[U_{AE \rightarrow BE'} (\varphi_A \otimes \omega_E) U_{AE \rightarrow BE'}^{\dagger} ]\]

where $U_{AE \rightarrow BE'}$ is a unitary matrix. Now, unitary matrices are square matrices implying that their application on a matrix preserves the matrix's dimensions. So, we have

\[dim(AE) = dim(BE')\]

$Tr_{_{E'}}(.)$ is the partial trace of $(.)$ with respect to matrix $E'$ defined as

\[ Tr_{_{B}}(X_{AB}) = \Sigma_i \langle i |_{B} X | i \rangle_{B} \]

where $\{ | i \rangle_{B} \}_i$ forms the orthonormal basis of $B$.

The partial trace with respect to $B$ trims the input $X_{AB}$ to a matrix $Y_A$ in $A$ by removing all indication of $B$.

\item $M_A \otimes N_B (\varphi_{AB})$  is a \emph{local operation} on $\varphi_{AB}$ where $M_A$ acts on $A$ and $N_B$ acts on $B$. Moreover, 

\[ [ M_A \otimes \mathcal{I}_B, \mathcal{I}_A \otimes N_B ] = 0 \]

which indicates that the order of application of $M_A$ and $N_B$ on $\varphi_{AB}$ doesn't matter.

\item We also note

\[ Tr_{_A}[M_A \otimes N_B (\varphi_{AB})] = N_B(\varphi_B) \]

where 

\[ \varphi_B = Tr_{_A}[\varphi_{AB}] \]

\end{description}

\end{document}
